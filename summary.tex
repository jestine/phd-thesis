

\section*{Executive Summary}

After considering a number of intermetallic systems, we concluded that a promising initial step in alloy design would be to explore (Cr, V)$_{solid}$ $_{solution}$--(Cr, V)$_3$Si$_{intermetallic}$ alloys.  These two-phase eutectic alloys have a solid-solution phase for fracture-toughness and an intermetallic phase for high-temperature strength.  They have low density, and have been designed to have no phase transformation at all operating temperatures, and no precipitation of undesirable additional phases.  Chromium had been chosen as a constituent to provide oxidation and corrosion resistance to temperatures of up to 900\celsius, and silicon had been chosen to provide oxidation resistance at temperatures above 900\celsius.  Fracture-toughness issues faced by chromium include its intrinsic brittleness at room temperature and its susceptibility to embrittlement due to interstitial impurities, especially that of nitrogen-embrittlement at high temperatures.  Vanadium is ductile at room temperature, and forms a perfect solid-solution with chromium.  It was hoped that alloys containing vanadium would demonstrate an increase in fracture-toughness, partly through the intrinsic character of vanadium, and partly through a decrease in the number of ordered bonds in the solid-solution.  V$_3$Si has been reported to display higher hardness than Cr$_3$Si at all temperatures up to 1200\celsius, and can contribute high-temperature strength to the alloy system.  Unfortunately it was found extremely difficult to melt alloys of the compositions selected with in-house facilities, but some samples were eventually melted in facilities at Warwick University and Birmingham University. Scanning electron microscopy and wavelength dispersive x-ray analysis were used to characterise the microstructures of these alloys, compression data at high temperatures was obtained, and elastic moduli and oxidation behaviour measured. The preliminary results showed that the  measured strength and oxidation behaviour of some of these alloys matched the expectations to some degree, but microanalysis showed that none of these samples was adequately melted to produce homogeneous alloys with optimised structures. Even if homogeneous alloys could have been prepared it was clear that the coarseness of the microstructures prepared from melting and casting limited their potential.  The main lesson learnt was that for comprehensive work to be carried out on these potential replacements for nickel-based superalloys, a range of very expensive facilities is required, which is currently not available. Until appropriate melting, atomisation, processing and testing facilities are available, it is suggested that future experimental work on such alloys should not be undertaken, but it is hoped that industry and government will make the necessary investment so that the work described in this thesis can be built upon.
