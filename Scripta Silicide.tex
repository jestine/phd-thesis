\documentclass[preprint]{elsarticle}

\usepackage{amsmath}
\usepackage{float}
\usepackage{array}
\usepackage{amssymb}
\usepackage[amssymb,thinqspace,thinspace]{SIunits}
\usepackage{esint}
\usepackage{graphicx}
\usepackage{eucal}
\usepackage{textcomp}
\usepackage{units}
\usepackage{fixltx2e}
\usepackage{setspace}
\usepackage{bm} 
\usepackage{pifont}
\usepackage{geometry}
\usepackage{hyperref}
\usepackage{natbib}      

\begin{document}
\begin{frontmatter}
% Title, authors and addresses
\title{Directionally-Solidified X-X$_3$Si Eutectics of Chromium and Vanadium}
%
\author[CAM]{J. Ang\corref{cor1}}
\ead{jestine.ang@gmail.com}
\author[GE]{V.A. Vorontsov}
\author[OSU]{K.A. Roberts}
\author[IMP]{C.A. Haywards}
\author[IMP]{H.J Stone}
\author[CAM]{C.M.F. Rae}
%\ead{cr18@cam.ac.uk}

\cortext[cor1]{Corresponding author}

\address[CAM]{Department of Materials Science and Metallurgy, University of Cambridge, Pembroke Street, Cambridge, CB2 3QZ, UK}
\address[GE]{GE Global Research, Niskayuna, NY, 12309, USA}
\address[OSU]{Department of Materials Science and Engineering, The Ohio State University, 2041 College Road, Columbus, OH 43210, USA}
\address[IMP]{Department of Materials, Royal School of Mines, Imperial College, South Kensington, London SW7 2AZ, UK}
%
% ABSTRACT
\begin{abstract}
An alternative high temperature structural alloy system based on the X-X$_3$Si eutectic compositions of chromium and vanadium is put forward. These low-density (~6g/cm$_3$) eutectics have a bcc solid-solution to increase alloy fracture toughness, and a high temperature load-bearing A15 X$_3$Si phase. Chromium silicides typically exhibit excellent corrosion and oxidation resistance, but are unacceptably brittle and retain insufficient high temperature strength \cite{jackson96, fleischer89, sadananda99}. Vanadium is tough, has perfect miscibility with chromium, and forms an A15 X$_3$Si with superior high temperature hardness than Cr$_3$Si [shah92]. 


\end{abstract}

% KEYWORDS
\begin{keyword}
Chromium Silicides, Vanadium Silicides, High Temperature Oxidation, Microstructure, Blade Alloy Design.
\end{keyword}
\end{frontmatter}
%
% MAIN TEXT
\section{Introduction}

The aim is to engineer a low-density gas-turbine blade alloy based on the Cr-Cr$_3$Si system that can operate at 1200�C/ 100MPa as a replacement for nickel-base superalloys. Higher Turbine Entry Temperatures can thus be achieved, and blade cooling systems can be minimised.
The material cost will be no more than �200/kg, which will be less than CMSX-4. The final alloy will form a dual-layer oxide with a Cr$_2$O$_3$ bottom layer to protect against corrosion, and a SiO$_2$ top layer to protect against high temperature oxidation. 

\section{Alloy Design}

The Cr-Cr$_3$Si eutectic has been chosen as the base alloy. Its eutectic temperature is 1705$\celsius$ \cite{gokhale90}, it has a solid solution component to maximise fracture toughness \cite{shah92}, and its density is a mere 6.81g/cm$^3$. Chromium suffers from impurity embrittlement, and has been shown to have inferior high temperature properties to the molybdenum and niobium silicides \cite{jackson96,fleischer89}.

The analogous, low density V-V$_3$Si system, with a eutectic temperature of 1870$\celsius$  \cite{smith90}. Vanadium will be the first alloying addition, substituted for Cr-Cr$_3$Si in 25at$\%$ increments. This could improve the toughness and strength of the solid solution \cite{shah92}. The superior hardness of V$_3$Si at 1200$\celsius$ to Cr$_3$Si \cite{jackson96} counterbalances the inability of vanadium to form a protective oxide. There are no ternary phase diagrams on Cr-V-Si available at present; however, it has been reported that Cr$_3$Si and V$_3$Si form a perfect solution \cite{shah92}, as do Cr and V \cite{gokhale90}. Precipitation of a tertiary phase is not anticipated in this simple ternary alloy system. 

\caption{Binary Phase Diagrams of Cr, V and Si.}

\begin{table*}
\onecolumn
\caption{Nominal Eutectic Compositions}
\label{tab:nominal}
\twocolumn
\begin{tabular*}{\textwidth}{ @{\extracolsep{\fill}} c c c c c c }
\end{tabular*}
\onecolumn
\end{table*}

\caption{Composition Map of Nominal Eutectic Compositions.}

\section{Alloy Manufacture}

Intermetallics currently experience a low degree of microstructural refinement, as many have very high melting points above 2000$\celsius$. The choice of casting equipment is limited for such materials. Thus, a significant portion of research has been performed on ingots that are arc-melted or processed by powder-metallurgy \cite{miracle94b,shah92,perepezko01,alur04}. The presence of grain boundaries in material manufactured using these methods is disadvantageous for high temperature creep resistance.

Microstructural refinement can be achieved with specimens prepared by directional solidification, as they will have substantially fewer grains and possess texture. Eutectic compositions have lower melting points than their constituent phases, which improves their casting feasibility. For eutectic compositions, low solidification rates allow a planar growth front to be maintained, allowing a neat lamellar structure to develop. Plapp and Karma have modelled eutectic planar front directional solidification evolving into eutectic colony formation due to the presence of a ternary impurity \cite{plapp02}, and illustrated the careful control of solidification conditions required for in-situ composite manufacture. 

\subsection{Non-Directionally Solidified Material}

All non-directionally solidified alloys were initially arc-melted. These ingots were turned over and re-melted 4 times. Despite this, extensive segregation in the 15-30g arc-melt specimens persisted. The melt is viscous and has very high surface tension, with the propensity for forming beads of melt instead of flowing to fit the mould holding it. Consequently, eighbouring grains can possess vastly different microstructures, with one having a lamellar microstructure, and its neighbour comprising mostly of dendrites. 

Plasma-melting at the Department of Metallurgy at the University of Birmingham was then pursued as alternative route of manufacture. Chips and small chunks of commercially pure elements were used as feedstock for the 2kg ingots. Each ingot was re-melted 4 times. Although the plasma melted stock maintained superior sample homogeniety, substantial unmelted stock material remained. Possible causes for this would be the high viscosity and surface tension induced by high silicon and chromium contents. Flecks of unmelted silicon were present to a depth of 2 cm in the alloy. As these ingots experienced lower rates of solidification, they possessed coarser microstructures than arc-melt ingots. 

All non-directionally solidified ingots are susceptible to cracking during solidification and machining. A few plasma-melted samples displayed catastrophic failure during high temperature 8-hour creep testing, which did not occur in the directionally solidified samples. The rapid solidification rates encountered in arc-melting and plasma melting must have led to thermal shock, resulting in the formation micro-cracks and cracks.


\subsection{Directionally Solidified Material}

In-house directional solidification in a radio frequency (RF) furnace is possible for materials with melting temperatures of up to 1800$\celsius$. This casting temperature is sufficient for manufacture of chromium-rich eutectics, but not for eutectics such as Ta-Ta$_3$Si, which melts at 2260$\celsius$  \cite{schlesinger94}, 

The V-V$_3$Si eutectic melts at 1870$\celsius$ , and was cast in a four-mirror xenon-lamp image furnace at the University of Warwick. Material manufacture using this technique has proven experimentally difficult \cite{shah92,bei03}. Feed ingot chemistry needs to be homogeneous, growth and rotation rates need to be controlled to prevent orientation misalignment. 

Arc-melted and plasma melted feed-stock was used in both routes. The high chromium compositions had very high surface tension and viscosities, making manufacture more difficult. In comparison, the compositions high in vanadium were less viscous, which improved sample homogeniety.


\section{Microstructure}

Cr-Cr$_3$Si forms long, flat lamellae under all manufacturing conditions. As chromium is substituted out for vanadium, lamellar spacing increases and the lamellae have a more equiaxed nature. Vanadium coarsens and destabilises the lamellae to a limited extent. This tendency may result in faster rates of spheroidisation and lamellae break-up during high temperature exposure \cite{english63}. Consequently, there would be a substantial loss in high temperature strength.


No tertiary precipitates have been found in the intermediate alloy compositions, which shows that chromium and vanadium are perfectly miscible, as are Cr$_3$Si and V$_3$Si. 
In off-eutectic compositions of Cr-Cr$_3$Si, a substantial adjacent envelope formed around the dendrites. In V-V$_3$Si, small dendritic arms formed thin non-lamellar envelopes around them when the local composition was off-eutectic. Melt viscosities may play a role in this behaviour. The addition of vanadium can improve ease of alloy manufacture. 


\section{Composition Analysis}

The microstructures were assessed by optical microscopy and SEM/EDS. Extensive wavelength dispersive spectroscopy (WDS) analysis was also performed, although for the purposes of this paper, only the eutectic compositions have been shown in Figure 3. Frequent calibration using iron and calcium silicates and pure element standards ensured consistent data, and only measurements with 100.0+/- .9 total weight percent were taken as acceptable. 

There was difficulty in obtaining reliable data despite repeated calibration with pure standards and silicates of calcium and iron, longer counting times, and re-defining the frequency ranges of data collected for each element to ensure no overlap of relevant data acquired. This difficulty could be due to electron interactions between those of silicon and the transition metals. The fine, dual-phase nature of these alloys will also interfere with data collection. Acquisition of secondary standards containing all elements can help eliminate this issue. 

A 15\micro\metre probe was used to take 5x5 to 10x10 grid measurements at 30\micro\metre intervals to determine the mean compositions of the 5 alloys. This has been plotted in Figure 3. The measured compositions of the two-element eutectics are very close to reported values \cite{gokhale90,smith90}. WDS measurements show the intermediate compositions having silicon contents similar to V-V$_3$Si and lower than Cr-Cr$_3$Si. In the intermediate alloys, the eutectics have a small composition range since there is a phase field where they can exist. Adding compatible elements are added to this system, it can help expand the eutectic phase field. There would be easier microstructural control over the manufacture of such specimens than with 2-element eutectics.

Closer observation of the X-X$_3$Si interface show that there is segregation of chromium present [figure 2]. Being the densest element, its segregation appears as light clusters fringing the interface under the back-scatter detector. This segregation is most extensive in arc-melted samples. Fast solidification rates could be partially accountable for this and thermal exposure should alleviate it.  
 

\section{Oxidation}

To ascertain the limit of vanadium substitution for chromium prior to runaway oxidation, a series of alloys within the tie-tetrahedron of the (Cr,V)-(Cr,V)$_3$Si system have been investigated following 10-hour isothermal exposures at 800�C and 1200$\celsius$. All specimens were polished down to a mirror finish, placed in their respective alumina crucibles and into a box furnace at their respective temperatures, then air-cooled in their alumina crucibles at the end of 10 hours.  

At 800$\celsius$, the Cr-Cr$_3$Si eutectic performs well, forming a green chromia layer. At 1200$\celsius$, most of its oxide spalls off upon cooling, as there was insufficient Si to form a protective SiO$_2$ layer. Chromium oxide is a dense adherent oxide of choice below 950$\celsius$, but is reported to be volatile at higher temperatures.
Formation of rust-coloured, oxygen permeable V$_2$O$_5$ has caused in runaway oxidation in the V-V$_3$Si eutectic at both temperatures. Presence of minute quantities of this oxide in the atmosphere has been reported to shorten the life of oxides on other alloys in an aero-engine[superalloys II?]. Thus, its emission from any feasible alloy should be minimised.

The Cr-rich intermediate compositions fared better than the pure binary eutectics; no oxide spallation or runaway oxidation was observed. Vaporisation of vanadium pentoxide was proportional to increasing vanadium content in the alloys, and (1/4Cr,3/4V)-(1/4,3/4)$_3$Si forms substantial vanadium pentoxide. 
SEM micrographs show the top surfaces of the oxides have many small grains of SiO$_2$ sitting among a layer of mixed chromium and vanadium oxides. This mixed oxide layer will have inferior oxidation resistance to a dual layer structure of chromium oxide and silica. With prolonged thermal exposure, the vanadium oxide in the mixed oxide layer will vaporise, leaving behind a weak, porous structure. Since the silicon content of these alloys is not sufficient to form a pure silica layer, further oxidation will occur.

Regardless, these preliminary observations indicate that a 25-50at\% vanadium substitution for chromium can suppress the extensive oxide spallation of Cr-Cr$_3$Si at 1200$\celsius$. An off-eutectic, Cr$_3$Si-rich alloy would have better oxidation resistance if it has sufficient silicon to form a complete layer of SiO$_2$ upon thermal exposure. 
Single-phase intermetallic alloys have been shown to have superior high temperature mechanical properties to dual-phase alloys with a solid solution phase \cite{shah92}. Since the resultant dendritic structure would have a larger volume fraction of the load-bearing silicide phase, one can expect it to be stronger at high temperature. Unfortunately, it would also be a magnitude coarser than the lamellar eutectic structure, and would make it less effective at crack deflection. The alloy's fracture toughness would thus decrease. 



\section{Conclusion}

A preliminary investigation into the microstructural and oxidative properties for the tie-tetrahedron of Cr-Cr$_3$Si and V-V$_3$Si has been conducted. 
Fine, flat lamellae are seen in the Cr-Cr$_3$Si eutectic. They coarsen and destabilise to a limited extent with the addition of vanadium, which may result in higher spheroidisation rates and lamellae break-up during high temperature exposure.

The intermediate compositions have silicon contents similar to V-V$_3$Si and lower than Cr-Cr$_3$Si. The lamellae are maintained over a 1at\% composition range for silicon. No tertiary precipitates have been found in the intermediate alloy compositions, which shows that chromium and vanadium are indeed perfectly miscible, as are Cr$_3$Si and V$_3$Si. 
10-hour isothermal oxidation indicate that a 25-50at\% vanadium substitution for chromium can suppress the extensive oxide spallation of Cr-Cr$_3$Si at 1200$\celsius$. An off-eutectic, Cr$_3$Si-rich alloy could prevent this spallation if it has sufficient silicon to form a complete layer of SiO$_2$ upon thermal exposure. Such an alloy with silicide dendrites may be substantially less tough compared to its eutectic counterpart. It would also have a coarser microstructure, and the ramifications on the alloy's high temperature mechanical properties of this is yet unknown.  


\section{Work-in-progress}

Current TEM work on the V-V$_3$Si eutectic show dislocation networks at interface between the A2 solid solution and the A15 intermetallic phase. This indicates that the interface is semi-coherent. The orientation relationship between the two phases, together with the lattice misfit, will be established once specimen preparation has been optimised. These misfit values will be compared to those measured in a neutron diffraction experiment at the ENGINX beamline at the Rutherford Appleton Laboratory, UK. 

The load-partitioning behaviour of these alloys is being studied at using a neutron source at ENGINX for alloys with high chromium content, and using a light source at ID15B at the European Synchrotron Radiation Facility (ESRF) for alloys with high vanadium content, as vanadium is neutron transparent.
From this preliminary evaluation, (�Cr,�V)-(�Cr,�V)$_3$Si has been chosen as the base alloy for the quaternary element additions being evaluated at present. High temperature strength has been found to have increased through solid solution strengthening and X$_3$Si reinforcement. Room temperature toughness and thermal shock resistance has also increased, and a lowering of melt viscosity has been achieved.

\section{Acknowledgements}

The efforts of Tim ? at the Department of Materials Science at the University of Birmingham in the manufacture of plasma melt ingots is appreciated.
The spark machining of plasma melt ingots by Electroform is acknowledged.

% BIBLIOGRAPHY
\bibliographystyle{elsart-num}
\bibliography{bibliography}
%

\begin{figure*}[t]
\begin{center}
\includegraphics[width=\textwidth]{figureplapp}
\end{center}
\caption{Break-down of the planar growth front due to a ternary impurity in a eutectic.} 
\label{fig:fig1}
\end{figure*}

\begin{figure*}[t]
\begin{center}
\includegraphics[width=\textwidth]{figuremicrostructure}
\end{center}
\caption{(i-v) Non-directionally Solidified and (vi-x) Directionally solidified microstructures of the (Cr,V)-(Cr,V)$_3$Si alloys} 
\label{fig:fig2}
\end{figure*}

\begin{figure*}[t]
\begin{center}
\includegraphics[width=\textwidth]{figurecompositionmap}
\end{center}
\caption{Composition map of eutectic alloys as measured by WDS} 
\label{fig:fig3}
\end{figure*}

\begin{figure*}[t]
\begin{center}
\includegraphics[width=\textwidth]{figureoxidation}
\end{center}
\caption{10-hour isothermal test of Cr-Cr$_3$Si, (3/4Cr,1/4V)-(3/4Cr,1/4V)$_3$Si, (1/2Cr,1/2V)-(1/2Cr,1/2V)$_3$Si (1/4Cr,3/4V)-(1/4Cr,3/4V)$_3$Si and V-V$_3$Si (i-v) at 800degC and (vi-x) at 1200$\celsius$. 
Top views and cross-sections of the oxide structures of Cr-Cr$_3$Si, (3/4Cr,1/4V)-(3/4Cr,1/4V)$_3$Si and (1/2Cr,1/2V)-(1/2Cr,1/2V)$_3$Si after 10 hours at 1200$\celsius$.} 
\label{fig:fig4}
\end{figure*}


\end{document}

